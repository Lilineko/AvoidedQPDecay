% ****** Start of file apssamp.tex ******
%
%   This file is part of the APS files in the REVTeX 4.1 distribution.
%   Version 4.1r of REVTeX, August 2010
%
%   Copyright (c) 2009, 2010 The American Physical Society.
%
%   See the REVTeX 4 README file for restrictions and more information.
%
% TeX'ing this file requires that you have AMS-LaTeX 2.0 installed
% as well as the rest of the prerequisites for REVTeX 4.1
%
% See the REVTeX 4 README file
% It also requires running BibTeX. The commands are as follows:
%
%  1)  latex apssamp.tex
%  2)  bibtex apssamp
%  3)  latex apssamp.tex
%  4)  latex apssamp.tex
%
\documentclass[%
 reprint,
%superscriptaddress,
%groupedaddress,
%unsortedaddress,
%runinaddress,
%frontmatterverbose, 
%preprint,
%showpacs,preprintnumbers,
%nofootinbib,
%nobibnotes,
%bibnotes,
 amsmath,amssymb,
 aps, onecolumn,
%%pra,
prl,
%rmp,
%prstab,
%prstper,
%floatfix,
]{revtex4-1}

\usepackage{graphicx}% Include figure files
\usepackage{dcolumn}% Align table columns on decimal point
\usepackage{bm}% bold math
\usepackage[colorlinks=true, citecolor=red, linkcolor=blue, urlcolor=black]{hyperref}% add hypertext capabilities
%\usepackage[mathlines]{lineno}% Enable numbering of text and display math
%\linenumbers\relax % Commence numbering lines

%\usepackage[showframe,%Uncomment any one of the following lines to test 
%%scale=0.7, marginratio={1:1, 2:3}, ignoreall,% default settings
%%text={7in,10in},centering,
%%margin=1.5in,
%%total={6.5in,8.75in}, top=1.2in, left=0.9in, includefoot,
%%height=10in,a5paper,hmargin={3cm,0.8in},
%]{geometry}

\renewcommand{\arraystretch}{1.5}

\usepackage{pgffor}

\usepackage{tikz}
\newcommand{\tikzcircle}[2][red,fill=red]{\tikz[baseline=-0.5ex]\draw[#1,radius=#2] (0,0) circle ;}%
\newcommand{\site}{\tikzcircle[fill=none]{2pt}}
\newcommand{\magnon}{\tikzcircle[fill=red]{2pt}}
\newcommand{\hole}{\tikzcircle[fill=blue]{2pt}}

\def\d{{\textit{d}}}
\newcommand*{\field}[1]{\mathbb{#1}}%

\newcommand{\bra}[1]{\langle#1\rvert}
\newcommand{\ket}[1]{\lvert#1\rangle}
\newcommand{\mean}[1]{\langle#1\rangle}

\newcommand{\Bra}[1]{\left\langle#1\left\rvert}
\newcommand{\Ket}[1]{\right\lvert#1\right\rangle}

\newcommand{\AFA}{\ket{\downarrow\uparrow\downarrow\uparrow\downarrow\uparrow\hdots}}
\newcommand{\AFB}{\ket{\uparrow\downarrow\uparrow\downarrow\uparrow\downarrow\hdots}}

\newcommand{\FMA}{\ket{\uparrow\uparrow\uparrow\uparrow\uparrow\uparrow\hdots}}
\newcommand{\FMB}{\ket{\downarrow\downarrow\downarrow\downarrow\downarrow\downarrow\hdots}}

\def\tj{{$t$--$J$}}
\def\tjz{{$t$--$J^z$}}
\def\tjlam{{$t$--$J$--$\lambda$}}
\def\square{{$\mathbb{Z}^2$}}

\begin{document}

%\preprint{APS/433-SAW}

\title{The fate of the spin polaron in the 1D antiferromagnet: \\ supplementary materials
}

\author{Piotr Wrzosek$^1$}
\email{Piotr.Wrzosek@fuw.edu.pl}
\author{Adam K\l{}osi\'nski$^1$}
\author{Yao Wang$^2$}
\author{Mona Berciu$^3$}
\author{Cli\`o Efthimia Agrapidis$^1$}
\author{Krzysztof Wohlfeld$^1$}
 
\affiliation{%
$^1$Institute of Theoretical Physics, Faculty of Physics, University of Warsaw, Pasteura 5, PL-02093 Warsaw, Poland
}%

\affiliation{%
$^2$Department of Physics and Astronomy, Clemson University, Clemson, South Carolina 29631, USA
}%

\affiliation{%
$^3$Dept. of Physics \& Astronomy, University of British Columbia, Vancouver, BC, Canada
and Quantum Matter Institute, University of British Columbia, Vancouver, BC, Canada
}%

\date{\today}% It is always \today, today,
             %  but any date may be explicitly specified
			 
\maketitle

\section{Rewriting the $t$--$J$ model in staggered field in the magnon-holon language}

The $t$--$J$ model Hamiltonian reads,
\begin{equation}
	H = -t\sum_{\mean{i,j}}\left(\tilde{c}_{i\sigma}^\dagger\tilde{c}_{j\sigma} + H.c\right)
	+ J\sum_{\mean{i,j}}\left[\frac{1}{2}\left(S_i^+S_j^- + S_i^-S_j^+\right) + S_i^zS_j^z - \frac{1}{4}\tilde{n}_i\tilde{n}_j\right],
	\label{eq:tJHam}
\end{equation}
where $\tilde{c}_{i\sigma}^\dagger = c_{i\sigma}^\dagger(1-n_{i\bar{\sigma}})$  creates an electron only if site $i$ is unoccupied. Let us consider a $t$--$J$ chain in between two Ising chains. This system is described by the Hamiltonian,
\begin{equation}
	H_{cc} = H + H_\parallel + H_\perp,
	\label{eq:SystemHam}
\end{equation}
%
with
%
\begin{equation}
	H_\parallel = J_\parallel \sum_{\mean{i,j}} \left( S_{i+\hat{y}}^z S_{j+\hat{y}}^z + S_{i-\hat{y}}^zS_{j-\hat{y}}^z \right),
	\label{eq:IsingHam}
\end{equation}
%
being the Ising Hamiltonian describing the two chains around the $t$--$J$ chain, and
%
\begin{equation}
	H_\perp = \frac{J_\perp}{2} \sum_{\mean{i,j}} \left( S_{i}^z S_{i+\hat{y}}^z + S_{j}^z S_{j+\hat{y}}^z + S_{i}^z S_{i-\hat{y}}^z + S_{j}^z S_{j-\hat{y}}^z \right),
	\label{eq:ICHam}
\end{equation}
%
being the interchain coupling Hamiltonian. In \eqref{eq:SystemHam}, $H$ is the $t$--$J$ Hamiltonian given in \eqref{eq:tJHam} with $J = J_\parallel$, $\hat y$ denotes which Ising chain is considered.

Assuming that the presence of the $t$--$J$ chain does not perturb the perfect Ising ordering in the ground state of \eqref{eq:IsingHam}, we obtain
\begin{align}
	H_\perp 
	&= \frac{J_\perp}{2} \sum_{\mean{i,j}} \left( S_{i}^z \mean{S_{i+\hat{y}}^z} + S_{j}^z \mean{S_{j+\hat{y}}^z} + S_{i}^z \mean{S_{i-\hat{y}}^z} + S_{j}^z \mean{S_{j-\hat{y}}^z} \right) \\
	&= \frac{J_\perp}{2} \sum_{\mean{i,j}} \left[(-1)^i S_{i}^z + (-1)^j S_{j}^z \right],
	\label{eq:StaggeredField}
\end{align}
where $\langle \cdot \rangle$ denote expectation values.
%
This corresponds to an effective staggered field $B = J_\perp$ acting on the $t$--$J$ chain. The field is induced by the coupling to the additional chains. Moreover, Eq. \eqref{eq:StaggeredField} is still valid in the case where more weakly coupled Ising chains are considered.\\

Now let us investigate how the additional staggered field looks like in the polaronic description already used in the main text. In order to introduce holes and magnons we start with a rotation of spins in one of the system's sublattices. This results in
%
\begin{equation}
	H_{\text{rot}} = -t\sum_{\mean{i,j}}\left(\tilde{c}_{i\sigma}^\dagger\tilde{c}_{j\bar{\sigma}} + H.c\right)
	+ J\sum_{\mean{i,j}}\left[\frac{1}{2}\left(S_i^+S_j^+ + S_i^-S_j^-\right) - S_i^zS_j^z - \frac{1}{4}\tilde{n}_i\tilde{n}_j\right].
\end{equation}
%
This allows for the introduction of holes and magnons according to the following transformations
%
\begin{equation}
	\begin{aligned}
	\tilde{c}_{i\uparrow}^\dag &= h_i, &\quad \tilde{c}_{i\uparrow} &= h_i^\dag (1 - a_i^\dag a_i), \\
	\tilde{c}_{i\downarrow}^\dag &= h_i a_i^\dag, &\quad \tilde{c}_{i\downarrow} &= h_i^\dag a_i,
	\end{aligned}
\end{equation}
%
\begin{equation}
	\begin{aligned}
		S_i^+ &= h_i h_i^\dag (1 - a_i^\dag a_i)a_i, &\quad S_i^z &= \left(\frac{1}{2} - a_i^\dag a_i \right) h_i h_i^\dag, \\
		S_i^- &= a_i^\dag (1 - a_i^\dag a_i) h_i h_i^\dag, &\quad \tilde{n}_i &= 1 - h_i^\dag h_i = h_i h_i^\dag,
	\end{aligned}
\end{equation}
where $a_i^\dag$ are bosonic creation operation at site $i$ denoting magnons and $h_i^\dag$ are fermionic creation operators at site $i$ denoting holons.
Here magnons can be understood as deviations from the state that has all the spins pointing up after the applied sublattice rotation. In the end, the model (up to a shift by a constant energy) reads:
%
\begin{align}
	\mathcal{H} &= \mathcal{H}_{t} + \mathcal{H}_{J},
\end{align}
%
where,
%	
\begin{equation}
	\begin{split}
	\mathcal{H}_{t} &= t \sum_{\mean{i,j}} \left[h_i^\dag h_j \left( a_i + a_j^\dag (1 -  a_i^\dag a_i) \right) + h_j^\dag h_i \left( a_j + a_i^\dag (1 -  a_j^\dag a_j) \right)\right],
	\end{split}
	\label{eq:ht}
\end{equation}
%
\begin{equation}
	\begin{aligned}
	\mathcal{H}_{J} &= \frac{J}{2}\sum_{\mean{i,j}} h_i h_i^\dag \left[(1 - a_i^\dag a_i)(1 - a_j^\dag a_j)a_i a_j + a_i^\dag a_j^\dag (1 - a_i^\dag a_i)(1 - a_j^\dag a_j) \right] h_j h_j^\dag \\
	&+ \frac{J}{2} \sum_{\mean{i,j}} h_i h_i^\dag \left[a_i^\dag a_i + a_j^\dag a_j - 2 a_i^\dag a_i a_j^\dag a_j - 1\right] h_j h_j^\dag.
	\end{aligned}
	\label{eq:hj}
\end{equation}

Now let us investigate the staggered magnetic field term given by
%
\begin{equation}
	H_{B} = \frac{B}{2} \sum_{\mean{i,j}} \left[(-1)^i S^z_i + (-1)^j S^z_j\right].
\end{equation}
%
Performing the same set of transformations we obtain (up to a constant energy shift),
%
\begin{equation}
	\mathcal{H}_{B} = \frac{B}{2} \sum_{\mean{i,j}} 
	\left[a_i^\dag a_i h_i h_i^\dag + a_j^\dag a_j h_j h_j^\dag \right] \approx \frac{B}{2} \sum_{\mean{i,j}} 
	h_i h_i^\dag \left[a_i^\dag a_i + a_j^\dag a_j \right] h_j h_j^\dag.
\end{equation}
%
The omitted terms on the right hand side of the approximation modify the magnetic field only around the hole and they are equal to $\frac{B}{2} \left[ a_i^\dag a_i h_j h_j^\dag + a_j^\dag a_j h_i h_i^\dag \right]$. In the end, we obtain
%
\begin{equation}
	\begin{aligned}
	\mathcal{H}_{J+B} &= \mathcal{H}_{J} + \mathcal{H}_{B} \\
	&\approx \frac{J}{2}\sum_{\mean{i,j}} h_i h_i^\dag \left[(1 - a_i^\dag a_i)(1 - a_j^\dag a_j)a_i a_j + a_i^\dag a_j^\dag (1 - a_i^\dag a_i)(1 - a_j^\dag a_j) \right] h_j h_j^\dag \\
	&+ \frac{J}{2} \sum_{\mean{i,j}} h_i h_i^\dag \left[\left(1+\frac{B}{J}\right)\left(a_i^\dag a_i + a_j^\dag a_j\right) - 2 a_i^\dag a_i a_j^\dag a_j - 1\right] h_j h_j^\dag.
	\end{aligned}
\end{equation}
%
Let us introduce a new coupling constant $J_z = J + B$ and a new parameter $\lambda = J / J_z$. Then, in the single hole limit, we can write
%
\begin{equation}
	\begin{aligned}
	\mathcal{H}_{J+B} 
	&\approx \frac{J_z \lambda}{2}\sum_{\mean{i,j}} h_i h_i^\dag \left[(1 - a_i^\dag a_i)(1 - a_j^\dag a_j)a_i a_j + a_i^\dag a_j^\dag (1 - a_i^\dag a_i)(1 - a_j^\dag a_j) \right] h_j h_j^\dag  \\
	&+ \frac{J_z}{2} \sum_{\mean{i,j}} h_i h_i^\dag \left[a_i^\dag a_i + a_j^\dag a_j - 2\lambda a_i^\dag a_i a_j^\dag a_j \right] h_j h_j^\dag.
	\end{aligned}
\end{equation}
%
For $\lambda \neq 1$ the coupling constants in the $xy$ plane and in $z$ direction are different. Thus, the final model can be understood as the $t$--$J$ model with XXZ anisotropy {\it and} rescaled magnon-magnon interaction.

Finally, we can relate the parameters of the coupled chains system to the parameters of the 1D $t$--$J$ model with rescaled magnon-magnon interactions and XXZ anisotropy by 
%
\begin{align}
	J_z &= J + B = J_\parallel + J_\perp, \\
	\lambda &= \frac{J}{J_z} = \frac{1}{1 + \frac{J_\perp}{J_\parallel}}.
\end{align}
%
In the TABLE~\ref{tab:params}. we present values used in calculations for Fig.~4b in the main text.

\section{SU(2) symmetry breaking in the $t$--$J$ model \\ with tuneable magnon-magnon interactions}

We start by re-expressing the magnon-magnon interaction term in the `standard' (i.e. spin) language,

\begin{equation}
    a_i^\dag a_i a_j^\dag a_j = -S_i^z S_j^z + \frac{1}{4}\tilde{n}_i\tilde{n}_j - \frac{1}{2}\left(\xi_i^\mathcal{A} S_i^z \ + \xi_j^\mathcal{A} S_j^z \right)\tilde{n}_i\tilde{n}_j,
\end{equation}
%
where $\xi_i^\mathcal{A}$ equals $-1$ for $i\in\mathcal{A}$ and $1$ otherwise, with $\mathcal{A},\mathcal{B}$ denoting the two sublattices of the bipartite lattice. Thus, Hamiltonian (2) of the main text (i.e. the $t$--$J$ model with tuneable magnon-magnon interactions) reads,
%
\begin{equation}
        \begin{aligned}
    	&H = -t\sum_{\mean{i,j}}\left(\tilde{c}_{i\sigma}^\dagger\tilde{c}_{j\sigma} + \text{H.c.}\right)
	+ J\sum_{\mean{i,j}}\left\{S_i S_j - \frac{1}{4}\tilde{n}_i\tilde{n}_j 
	+ \left(\lambda-1\right) \left[S_i^z S_j^z - \frac{1}{4}\tilde{n}_i\tilde{n}_j + \frac{1}{2}\left(\xi_i^\mathcal{A} S_i^z \ + \xi_j^\mathcal{A} S_j^z \right)\tilde{n}_i\tilde{n}_j\right] \right\}.
	\end{aligned}
	\label{eq:lambda_spin_model}
\end{equation}
In the above Hamiltonian \eqref{eq:lambda_spin_model}, the term
%
\begin{equation}
    \frac{1}{2}\left(\xi_i^\mathcal{A} S_i^z \ + \xi_j^\mathcal{A} S_j^z \right)\tilde{n}_i\tilde{n}_j
        \label{eq:staggered_term}
\end{equation}
%
can be understood as a staggered field acting on all spins although it is halved for the neighbors of the hole. This term contributes to the Hamiltonian once $\lambda \neq 1$ and explicitly breaks the SU(2) symmetry.  
\\

\begin{table}[h]
	\begin{center}
	\begin{tabular}{|| c | c || c | c ||} 
		\hline
		~~$J_\parallel$~~ & ~~$J_\perp$~~ & ~~$J_z$~~ & ~~$\lambda$~~ \\
		\hline\hline
		~~$0.4t$~~ & ~~$0.004t$~~ & ~~$0.404t$~~ & ~~$\frac{100}{101}$~~ \\  
		\hline
		$0.4t$ & $0.04t$ & $0.44t$ & $\frac{10}{11}$ \\ 
		\hline
		$0.4t$ & $0.2t$ & $0.6t$ & $\frac{2}{3}$ \\
		\hline
	\end{tabular}
	\end{center}
	\caption{Table presenting various values of parameters equivalent in coupled chains problem and $t$--$J$ model with rescaled magnon-magnon interactions and XXZ anisotropy.}
	\label{tab:params}
\end{table}

\bibliographystyle{apsrev4-1}
\bibliography{saw}

\end{document}
