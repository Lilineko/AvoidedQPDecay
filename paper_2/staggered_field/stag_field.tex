\documentclass[12pt, a4paper]{article}
\usepackage[a4paper,margin=1in,footskip=0.25in]{geometry}
 
\usepackage{graphicx}
\usepackage{enumerate}
\usepackage{amsmath}
\usepackage{amssymb}
\usepackage[utf8]{inputenc}
\usepackage{physics}
\usepackage{xcolor}
\usepackage{hyperref}
\usepackage{subfig}
		
\newcommand{\mean}[1]{\langle#1\rangle}

\begin{document}

\section{The~Model}
The $t$-$J$ model Hamiltonian reads,
\begin{equation}
	H = -t\sum_{\mean{i,j}}\left(\tilde{c}_{i\sigma}^\dagger\tilde{c}_{j\sigma} + H.c\right)
	+ J\sum_{\mean{i,j}}\left(\frac{1}{2}\left[S_i^+S_j^- + S_i^-S_j^+\right] + S_i^zS_j^z - \frac{1}{4}\tilde{n}_i\tilde{n}_j\right),
\end{equation}
where $\tilde{c}_{i\sigma}^\dagger = c_{i\sigma}^\dagger(1-n_{i\bar{\sigma}})$ can create electrons only on unoccupied sites. In order to introduce holes and magnons we start with rotation of spins in one sublattice of the system. This results in
\begin{equation}
	H_{\text{rot}} = -t\sum_{\mean{i,j}}\left(\tilde{c}_{i\sigma}^\dagger\tilde{c}_{j\bar{\sigma}} + H.c\right)
	+ J\sum_{\mean{i,j}}\left(\frac{1}{2}\left[S_i^+S_j^+ + S_i^-S_j^-\right] - S_i^zS_j^z - \frac{1}{4}\tilde{n}_i\tilde{n}_j\right).
\end{equation}
This allows for introduction of holes and magnons accoridng to the following transformations
\begin{equation}
	\begin{aligned}
	\tilde{c}_{i\uparrow}^\dag &= h_i, &\quad \tilde{c}_{i\uparrow} &= h_i^\dag (1 - a_i^\dag a_i), \\
	\tilde{c}_{i\downarrow}^\dag &= h_i a_i^\dag, &\quad \tilde{c}_{i\downarrow} &= h_i^\dag a_i,
	\end{aligned}
\end{equation}
\begin{equation}
	\begin{aligned}
		S_i^+ &= h_i h_i^\dag (1 - a_i^\dag a_i)a_i, &\quad S_i^z &= (\frac{1}{2} - a_i^\dag a_i) h_i h_i^\dag, \\
		S_i^- &= a_i^\dag (1 - a_i^\dag a_i) h_i h_i^\dag, &\quad \tilde{n}_i &= 1 - h_i^\dag h_i = h_i h_i^\dag.
	\end{aligned}
\end{equation}
Here magnons can be understood as deviations from state that after the rotation has all the spins pointing up. In the end the model (up to the shift by a constant) reads,
\begin{align}
	\mathcal{H} &= \mathcal{H}_{t} + \mathcal{H}_{J},
\end{align}
	where,
\begin{equation}
	\begin{split}
	\mathcal{H}_{t} &= t \sum_{\mean{i,j}} \left[h_i^\dag h_j \left( a_i + a_j^\dag (1 -  a_i^\dag a_i) \right) + h_j^\dag h_i \left( a_j + a_i^\dag (1 -  a_j^\dag a_j) \right)\right],
	\end{split}
	\label{eq:ht}
\end{equation}

\begin{equation}
	\begin{aligned}
	\mathcal{H}_{J} &= \frac{J}{2}\sum_{\mean{i,j}} h_i h_i^\dag \left[(1 - a_i^\dag a_i)(1 - a_j^\dag a_j)a_i a_j + a_i^\dag a_j^\dag (1 - a_i^\dag a_i)(1 - a_j^\dag a_j) \right] h_j h_j^\dag \\
	&+ \frac{J}{2} \sum_{\mean{i,j}} h_i h_i^\dag \left[a_i^\dag a_i + a_j^\dag a_j - 2 a_i^\dag a_i a_j^\dag a_j - 1\right] h_j h_j^\dag.
	\end{aligned}
	\label{eq:hj}
\end{equation}
Now let consider a staggered magnetic field,
\begin{equation}
	\mathcal{H}_{B} = \frac{B}{2} \sum_{\mean{i,j}} \left[(-1)^i S^z_i + (-1)^j S^z_j\right].
\end{equation}
Performing the same set of transformations as to the Hamiltonian we obtain (up to the constant energy shift),
\begin{equation}
	\mathcal{H}_{B} = \frac{B}{2} \sum_{\mean{i,j}} 
	\left[a_i^\dag a_i h_i h_i^\dag + a_j^\dag a_j h_j h_j^\dag \right] \approx \frac{B}{2} \sum_{\mean{i,j}} 
	h_i h_i^\dag \left[a_i^\dag a_i + a_j^\dag a_j \right] h_j h_j^\dag,
\end{equation}
where ommited terms on the righthandside of the approximation sign are modifying the magnetic field only around the hole and they are equal to $\frac{B}{2} \left[ a_i^\dag a_i h_j h_j^\dag + a_j^\dag a_j h_i h_i^\dag \right]$. In the end,
\begin{equation}
	\begin{aligned}
	\mathcal{H}_{J+B} &= \mathcal{H}_{J} + \mathcal{H}_{B} \approx \\
	&= \frac{J}{2}\sum_{\mean{i,j}} h_i h_i^\dag \left[(1 - a_i^\dag a_i)(1 - a_j^\dag a_j)a_i a_j + a_i^\dag a_j^\dag (1 - a_i^\dag a_i)(1 - a_j^\dag a_j) \right] h_j h_j^\dag \\
	&+ \frac{J}{2} \sum_{\mean{i,j}} h_i h_i^\dag \left[\left(1+\frac{B}{J}\right)\left(a_i^\dag a_i + a_j^\dag a_j\right) - 2 a_i^\dag a_i a_j^\dag a_j - 1\right] h_j h_j^\dag.
	\end{aligned}
\end{equation}
Let introduce new coupling constant $J_s = J + B$ and parameter $\lambda = J / J_s$. Then, in the limit of a single hole, we can write,
\begin{equation}
	\begin{aligned}
	\mathcal{H}_{J+B} 
	&\approx \frac{J_s \lambda}{2}\sum_{\mean{i,j}} h_i h_i^\dag \left[(1 - a_i^\dag a_i)(1 - a_j^\dag a_j)a_i a_j + a_i^\dag a_j^\dag (1 - a_i^\dag a_i)(1 - a_j^\dag a_j) \right] h_j h_j^\dag  \\
	&+ \frac{J_s}{2} \sum_{\mean{i,j}} h_i h_i^\dag \left[a_i^\dag a_i + a_j^\dag a_j - 2\lambda a_i^\dag a_i a_j^\dag a_j \right] h_j h_j^\dag.
	\end{aligned}
\end{equation}
For $\lambda \neq 1$ coupling constants in $xy$ plane and in $z$ direction are different. In the end the resulting model can be undestood as XXZ model with rescaled magnon-magnon interaction.

\bibliographystyle{apsrev4-1}
\bibliography{xxz}

\end{document}